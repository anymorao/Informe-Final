
%Portada del Documento
\chapter{Python}
\setlength{\unitlength}{1 cm} %Especificar unidad de trabajo
\thispagestyle{empty}
\begin{picture}(18,10)
\put(1.5,1){\includegraphics[width=10cm,height=8cm]{./imagenes4/apli1.png}}
\end{picture}



% fin de la portada


\newpage
\section{ Introducción}

El proyecto realizado con la finalidad de ayudar a personas con discapacidad visual, Interactuando con el usuario por medio de sonido y de instrucciones de voz.
 Consiste en narrar una aventura en la cual nosotros mismos le damos el final dependiendo de las opciones que escojamos,todo esto atravez de instruciones de voz. 
    \subsection{ Requisitos}
\begin{itemize}
\item Instalar Python 2.7.
 \item Instalar la libreria Pygame.
 \item Instalar la libreria Pyaudio.
\item Netbeans 6.9 IDE (no es necesario).
\item Instalar el plugins de Python para netbeans.

\end{itemize}


\begin{figure}[ht!]
 
   \centering
   %%----primera subfigura----
   \subfloat[]{
        \label{fig:pantalla:1}         %% Etiqueta para la primera subfigura
        \includegraphics[width=0.42\textwidth]{./imagenes4/img1.png}}
   \hspace{0.1\linewidth}
   %%----segunda subfigura----
   \subfloat[]{
        \label{fig:pantalla:2}         %% Etiqueta para la segunda subfigura
        \includegraphics[width=0.42\textwidth]{./imagenes4/img2.png}}\\[20pt]
   %%----tercera subfigura----
   \subfloat[]{
        \label{fig:pantalla:3}         %% Etiqueta para la tercera subfigura
        \includegraphics[width=0.42\textwidth]{./imagenes4/dibujo.png}}
\hspace{0.1\linewidth}
 %%----cuarta subfigura----
   \subfloat[]{
        \label{fig:pantalla:4}         %% Etiqueta para la tercera subfigura
        \includegraphics[width=0.42\textwidth]{./imagenes4/img3.png}}
    
\end{figure}


\section{ Como ejecutar el proyecto}
Primero copiamos el archivo del proyecto en cualquier directorio, luego abrimos NetBeans , seguimos los siguientes pasos:
\begin{enumerate}
\item Dar click Archivo -> Abrir Proyecto y buscamos la carpeta del proyecto en el directorio que lo guardamos.
\item Luego en la barra de herramientas dar click en ejecutar.
\end{enumerate}
     
\begin{figure}[ht!]
 
   \centering
   %%----primera subfigura----
   \subfloat[]{
        \label{fig:pantalla:1}         %% Etiqueta para la primera subfigura
        \includegraphics[width=0.42\textwidth]{./imagenes4/img4.png}}
   \hspace{0.1\linewidth}
   %%----segunda subfigura----
   \subfloat[]{
        \label{fig:pantalla:2}         %% Etiqueta para la segunda subfigura
        \includegraphics[width=0.42\textwidth]{./imagenes4/img5.png}}
\hspace{0.1\linewidth}
   %%----tercera subfigura----
   \subfloat[]{
        \label{fig:pantalla:3}         %% Etiqueta para la tercera subfigura
        \includegraphics[width=0.42\textwidth]{./imagenes4/apli.png}}
\end{figure}

\section{ Codigo}
 \begin{verbatim}
   
import reproducir
import consola_io

def main():
        
        # creamos la ventana y le indicamos un titulo:
       
	#empieza el proyecto narrando una introduccion
	#reproducir.reproduce("1a.wav")
	#reproducir.limpiar()

	#Inicia el cuento y se espera que el usuario ingrese un numero
	reproducir.reproduce("1a.wav")

	#validacion de un numero
	t=1
	while t:
               
		 #num2 = raw_input("Escoja una opcion: 1,2,3,4 o 9 \n Su eleccion es: ")
		print("Escoja una opcion: 1,2,3,4 o 9 \n  ")
		num2=consola_io.getkey()
		print("Su eleccion es: "+num2)
		try:
			num2=int(num2)
			if(num2==9):
				main()
			elif(num2==1):
				reproducir.reproduce("2.wav")
				t2=1
				while t2:
					#num3 = raw_input("Escoja una opcion: 1,2,3,4,5 o 9 \n Su eleccion es: ")
					print("Escoja una opcion: 1,2,3,4,5 o 9 \n ")					
					num3=consola_io.getkey()
					print("Su eleccion es: "+num3)
					try:
						num3=int(num3)
						if(num3==9):
							main()
						elif(num3==1):
							reproducir.reproduce("6.wav")
							t2=0
						elif(num3==2):
							reproducir.reproduce("7.wav")
							t2=0
						elif(num3==3):		
							reproducir.reproduce("8.wav")
							t2=0
						elif(num3==4):
							reproducir.reproduce("9.wav")
							t2=0
						elif(num3==5):
							reproducir.reproduce("10.wav")
							t2=0
								
					except ValueError:
						pass
				t=0

			elif(num2==2):
				reproducir.reproduce("3.wav")
				t3=1
				while t3:
					#num4 = raw_input("Escoja una opcion: 1,2 o 9 \n Su eleccion es:  ")
					print("Escoja una opcion: 1,2 o 9 \n ")
					num4=consola_io.getkey()
					print("Su eleccion es: "+num4)
					try:
						num4=int(num4)
						if(num4==9):
							main()
						elif(num4==1):
							reproducir.reproduce("11.wav")
							t3=0
							t4=1
							while t4:
								#num5 = raw_input("Escoja una opcion: 1,2 o 9 \n Su eleccion es: ")
								print("Escoja una opcion: 1,2 o 9 \n ")
								num5=consola_io.getkey()
								print("Su eleccion es: "+num5)
								try:
									num5=int(num5)
									if(num5==9):
										main()
									elif(num5==1):
										reproducir.reproduce("12.wav")
										t4=0
									elif(num5==2):
										reproducir.reproduce("13.wav")
										t4=0
										t5=1
										while t5:
											#num6 = raw_input("Escoja una opcion: 1,2 o 9 \n Su eleccion es: ")
											print("Escoja una opcion: 1,2 o 9 \n ")
											num6=consola_io.getkey()
											print("Su eleccion es: "+num6)
											try:
												num6=int(num6)
												if(num6==9):
													main()
												elif(num6==1):
													reproducir.reproduce("14.wav")
													t5=0
												elif(num6==2):
													reproducir.reproduce("15.wav")
													t5=0
											except ValueError:
												pass


								except ValueError:
									pass


						elif(num4==2):
							reproducir.reproduce("7.wav")
							t3=0
					except ValueError:
						pass
				t=0
				
			elif(num2==3):
				reproducir.reproduce("4.wav")
				t=0
			elif(num2==4):		
				reproducir.reproduce("5.wav")
				main()
		
		except ValueError:
			pass
	print "\nY asi \ncolorin colorado \neste cuento se ha acabado"

#ejemplo limpiar pantalla: reproducir:limpiar 
if __name__ == "__main__":
            main()
 \end{verbatim}

\section{ Experiencias}
\begin{itemize}
\item Python fue un lenguaje nuevo para mi pero debido a toda la informaciòn que hay en internet y a los ejemplos que encontre se me hizo facil. 
\item Solo se uso la libreria de audio para trabajar el proyecto el cual fue divertido realizarlo.  
\item Buscar el libro y grabar la historia fue interesante, tuve que leer muchas historias para ver cual era la mejor para el proyecto.
\item Hubo una historia a que me gusto mucho que se llama " El Misterio de las Piedras Sagradas ", esta historia es muy interesante pero a mi compañero le parecio muy larga asi que tuvimos que buscar otra.
\item La historia que grabamos la encontro mi compañero de proyecto en internet, es divertida y no muy laraga por eso decidimos hacer esa.
 \section{ Conclusiones} 
 Aunque Python se enfoca mas al desarrollo de juegos con la libreria mas usada que es pygame nuestro proyecto se enfoco en la parte de audio usando la libreria pyaudio la cual fue una experiencia interesante ya que nuestro proyecto podria mejorarse para utilizarlo como ayuda para personas con discapacidad visual.
\end{itemize}


